\documentclass[12pt]{article}
% \documentclass[12pt]{beamer}
\usepackage{amsmath,amsfonts,amssymb,amsthm}
\usepackage{color}
\usepackage[square,sort,comma,numbers]{natbib}
\usepackage[T1]{fontenc}
\usepackage{mathtools}
\usepackage{graphicx}
\usepackage{subcaption}
\usepackage{kotex}
\usepackage{bm}
\usepackage[shortlabels]{enumitem}
\usepackage{tcolorbox}
\usepackage{mathrsfs}
\usepackage{newtxtext}
\usepackage[lite,nofontinfo,zswash,straightbraces]{mtpro2}
\usepackage{setspace} %줄간격 package
%-------------------------------------------------------------------------------------
% Page layout
\addtolength{\textwidth}{1.2in}
\addtolength{\oddsidemargin}{-0.6in}
\addtolength{\textheight}{2.0in}
\addtolength{\topmargin}{-1.0in}
\renewcommand \baselinestretch{1.2}
\parskip = 6pt
% New command
\newcommand {\AND}{\quad\mbox{and}\quad}
\newcommand {\for}{\quad\mbox{ for }}
\newcommand {\half}{{\tfrac 12}}
% \newcommand {\bm} [1] {\mbox{\boldmath{$#1$}}}
\newcommand{\bs}{\boldsymbol}
\DeclareMathOperator{\Tr}{Tr}
\DeclareMathOperator{\Cov}{Cov}
\DeclareMathOperator{\Var}{Var}
\DeclareMathOperator{\E}{\mathbf{E}}
\DeclareMathOperator{\vvv}{vec}
\DeclareMathOperator{\vvvh}{vech}
% \setmainfont{TEX Gyre Termes}
% \setmathfont[bold-style=TeX]{TG Termes Math}
\begin{document}
\setstretch{1.4} %줄간격 조정
\title{SVI for SUR}
%------------------------------------------------------------------------------------%
%------------------------------------------------------------------------------------%
% \setcounter{section}{1}

\section{Wishart Distribution}
  a $p\times p$ Wishart random variate has a density of the form
  \begin{align}
    \mathcal{W}\left(\Omega\,|\,V,k\right) &= C(k,V)|\Omega|^{(k-p-1)/2}\exp\left(-\dfrac{1}{2}\Tr\left(V^{-1}\Omega\right)\right)\\
    C(k,V) &= |V|^{-k/2}\left(2^{kp/2}\pi^{p(p-1)/4}\prod_{i=1}^{p}\Gamma\left(\dfrac{k+1-i}{2}\right)\right)^{-1}
  \end{align}
  The log-density is
  \begin{align}
    \log\mathcal{W}\left(\Omega\,|\,k,V\right) &= \dfrac{k-p-1}{2}\log|\Omega|-\dfrac{1}{2}\Tr\left(V^{-1}\Omega\right)-\dfrac{k}{2}\log|V|\\
    &\quad -\dfrac{kp}{2}\log 2-\dfrac{p(p-1)}{4}\log \pi -\sum_{i=1}^{p}\log\Gamma\left(\dfrac{k+1-i}{2}\right)
  \end{align}
  Before jumping into the derivative of the log-density, let's do it by parts.
  \begin{itemize}
    \item $\ell=\Tr\left(V^{-1}\Omega\right)=\vvv(\Omega)'\vvv\left(V^{-1}\right)$
    \begin{align}
      d\ell &= \vvv(\Omega)'\vvv\left(dV^{-1}\right)\\
            &= \vvv(\Omega)'\vvv\left(-V^{-1}\,dV\,V^{-1}\right)\\
            &= -\vvv(\Omega)'\left(V^{-1}\otimes V^{-1}\right)D_{p}\,d\vvvh\left(V\right)\\
      \dfrac{d \Tr\left(V^{-1}\Omega\right)}{d\vvvh(V)'} &= -D_{p}'\left(V^{-1}\otimes V^{-1}\right)D_{p}\vvvh(\Omega)
    \end{align}
    where $D_{p}$ is the unique duplication matrix such that $D_{p}\vvvh(A)=\vvv(A)$.
    \item $\ell=\log|V|$
    \begin{align}
      \dfrac{d\log|V|}{d\vvvh(V)'} &= D_{p}\vvvh\left(V^{-1}\right)
    \end{align}
  \end{itemize}
  Therefore,
  \begin{align}
    \nabla_{\vvvh(V)}\log\mathcal{W}(\Omega\,|\,k,V) &= \dfrac{1}{2}D_{p}'\left(V^{-1}\otimes V^{-1}\right)D_{p}\vvvh(\Omega)-\dfrac{k}{2}\vvvh\left(V^{-1}\right)\\
    \nabla_{k}\log\mathcal{W}(\Omega\,|\,k,V) &= \dfrac{1}{2}\log|\Omega|-\dfrac{1}{2}\log|V|-\dfrac{p}{2}\log 2-\dfrac{1}{2}\sum_{i=1}^{p}\psi\left(\dfrac{k+1-i}{2}\right)
  \end{align}
\subsection{Fisher Information of Wishart}
To get the Fisher information matrix, we need to go through quite a few steps. First, according to \cite{MultivariateStat}, 
\begin{itemize}
  \item ($\Var(\vvv(\Omega))$)
  \begin{align}
    \Var(\vvv(\Omega)) &= k\left(\mathbf{I}_{p^{2}}+K_{pp}\right)(V\otimes V)
  \end{align}
  where $K_{pp}$ is a $p^{2}\times p^{2}$ commutation matrix such that
  \begin{equation}
    K_{pp}\vvv(C) = \vvv(C')
  \end{equation}
  for a $p\times p$ matrix $C$.
  \item ($\Var(\log|\Omega|)$)
  To get the variance of the log-determinant, we will rely on the following relation.
  \begin{equation}
    \dfrac{|\Omega|}{|V|} = \chi_{k}^{2}\chi_{k-1}^{2}\cdots\chi_{k-p+1}^{2}
  \end{equation}
  where every chi-squared random variables are independent of each other. We need to do variable transformation to get the density of log chi-squared random variate. If we say $X\sim \log\chi_{\nu}^{2}$, the density is
  \begin{equation}
    p(x) = \left(2^{\nu/2}\Gamma(\nu/2)\right)^{-1}\exp\left(\dfrac{1}{2}\nu x-\dfrac{1}{2}\exp(x)\right),\qquad -\infty<x<\infty
  \end{equation}
  Then, performing the integration, we obtain the following central moments
  \begin{align}
    \E(X) &= \log 2+\psi(\nu/2)\\
    \Var(X) &= \psi_{1}(\nu/2)
  \end{align}
  where $\psi_{1}$ is the tri-gamma function. Thus, since $\log|\Omega|-\log|V|=\sum_{i=1}^{p}\log \chi_{k-i+1}^{2}$,
  \begin{equation}
    \Var(\log|\Omega|) = \sum_{i=1}^{p}\psi_{1}\left(\dfrac{k-i+1}{2}\right)
  \end{equation}
  \item The block-diagonal matrices of the Fisher information have been obtained in the above items. However, it is quite difficult to compute the following off-diagonal covariance term:
  \begin{equation}
    \Cov(\vvv(\Omega),\log|\Omega|)
  \end{equation}
\end{itemize}
\section{SUR}
Seemingly Unrelated Regression model is constructed as follows.
\begin{equation}
  \mathbf{y}_{t} = X_{i}'\beta+\mathbf{e}_{t},\quad \mathbf{e}_{t}\sim\mathcal{N}\left(\mathbf{0},\Omega^{-1}\right)
\end{equation}
\begin{itemize}
\item $\beta\sim\mathcal{N}\left(\mu_{\beta}^{0},\Sigma_{\beta}^{0}\right),\quad (m\times 1)$
\item $\Omega\sim\mathcal{W}(k,V),\quad (p\times p)$
\end{itemize}
The varational posteriors are
\begin{itemize}
  \item $q(\beta) = \mathcal{N}\left(\mu_{\beta}^{q},\Sigma_{\beta}^{q}\right)$
  \item $q(\Omega) = \mathcal{W}\left(k_{q},V_{q}\right)$
\end{itemize}
Therefore,
\begin{align}
  \log h(\theta) &= \dfrac{T}{2}\log\det\Omega-\dfrac{1}{2}\sum_{t=1}^{T}\left(\mathbf{y}_{t}-X_{t}'\beta\right)'\Omega\left(\mathbf{y}_{t}-X_{t}'\beta\right)-\dfrac{Tp}{2}\log(2\pi)\\
  &\quad -\dfrac{1}{2}\log\det\Sigma_{\beta}^{0}-\dfrac{1}{2}\left(\beta-\mu_{\beta}^{0}\right)'{\Sigma_{\beta}^{0}}^{-1}\left(\beta-\mu_{\beta}^{0}\right)\\
  &\quad +\dfrac{k-p-1}{2}\log\det\Omega-\dfrac{1}{2}\Tr\left(V^{-1}\Omega\right)-\dfrac{k}{2}\log\det V-\dfrac{kp}{2}\log 2\\
  &\quad -\sum_{i=1}^{p}\log\Gamma\left(\dfrac{k+1-i}{2}\right)\\
  \log q_{\lambda}(\theta) &= -\dfrac{1}{2}\log\det \Sigma_{\beta}^{q}-\dfrac{1}{2}\left(\beta-\mu_{\beta}^{q}\right)'{\Sigma_{\beta}^{q}}^{-1}\left(\beta-\mu_{\beta}^{q}\right)\\
  &\quad + \dfrac{k_{q}-p-1}{2}\log\det\Omega-\dfrac{1}{2}\Tr\left(V_{q}^{-1}\Omega\right)-\dfrac{k_{q}}{2}\log\det V_{q}-\dfrac{k_{q}p}{2}\log 2\\
  &\quad -\sum_{i=1}^{p}\log\Gamma\left(\dfrac{k_{q}+1-i}{2}\right)
\end{align}
where $\theta=(\beta,\Omega)$ and $\lambda=\left(\mu_{\beta}^{q},\Sigma_{\beta}^{q},k_{q},V_{q}\right)$.
\newpage
\nocite{*}
\bibliographystyle{acm}
\bibliography{SVI_SUR}
\end{document}