\documentclass[11pt]{article}

% This first part of the file is called the PREAMBLE. It includes
% customizations and command definitions. The preamble is everything
% between \documentclass and \begin{document}.

\usepackage[margin=1in]{geometry} % set the margins to 1in on all sides
\usepackage{graphicx} % to include figures
\usepackage{amsmath} % great math stuff
\usepackage{amsfonts} % for blackboard bold, etc
\usepackage{amsthm} % better theorem environments
% various theorems, numbered by section
\usepackage{amssymb}
\usepackage[utf8]{inputenc}
\usepackage{booktabs}
\usepackage{array}
\usepackage{courier}
\usepackage[usenames, dvipsnames]{color}
\usepackage{titlesec}
\usepackage{empheq}
 
% Command "alignedbox{}{}" for a box within an align environment
% Source: http://www.latex-community.org/forum/viewtopic.php?f=46&t=8144
\newlength\dlf  % Define a new measure, dlf
\newcommand\alignedbox[2]{
% Argument #1 = before & if there were no box (lhs)
% Argument #2 = after & if there were no box (rhs)
&  % Alignment sign of the line
{
\settowidth\dlf{$\displaystyle #1$}  
    % The width of \dlf is the width of the lhs, with a displaystyle font
\addtolength\dlf{\fboxsep+\fboxrule}  
    % Add to it the distance to the box, and the width of the line of the box
\hspace{-\dlf}  
    % Move everything dlf units to the left, so that & #1 #2 is aligned under #1 & #2
\boxed{#1 #2}
    % Put a box around lhs and rhs
}
}


\newtheorem{thm}{Theorem}[section]
\newtheorem{lem}[thm]{Lemma}
\newtheorem{prop}[thm]{Proposition}
\newtheorem{cor}[thm]{Corollary}
\newtheorem{conj}[thm]{Conjecture}

\setcounter{secnumdepth}{4}

\titleformat{\paragraph}
{\normalfont\normalsize\bfseries}{\theparagraph}{1em}{}
\titlespacing*{\paragraph}
{0pt}{3.25ex plus 1ex minus .2ex}{1.5ex plus .2ex}

\definecolor{myblue}{RGB}{72, 165, 226}
\definecolor{myorange}{RGB}{222, 141, 8}

\setlength{\heavyrulewidth}{1.5pt}
\setlength{\abovetopsep}{4pt}


\DeclareMathOperator{\id}{id}
\DeclareMathOperator{\argmin}{\arg\!\min}
\DeclareMathOperator{\Tr}{Tr}

\newcommand{\bd}[1]{\mathbf{#1}} % for bolding symbols
\newcommand{\RR}{\mathbb{R}} % for Real numbers
\newcommand{\ZZ}{\mathbb{Z}} % for Integers
\newcommand{\col}[1]{\left[\begin{matrix} #1 \end{matrix} \right]}
\newcommand{\comb}[2]{\binom{#1^2 + #2^2}{#1+#2}}
\newcommand{\bs}{\boldsymbol}
\newcommand{\opn}{\operatorname}
\begin{document}
\nocite{*}

\title{Incorporating Random effect into GP Sparse Approximation}

\author{Daeyoung Lim\thanks{Prof. Taeryon Choi} \\
Department of Statistics \\
Korea University}

\maketitle

\section{GP linear mixed model}
\subsection{Model specifications}
The model specifications remain the same except that the random effect term has been added.
\begin{align*}
  y|\theta &\sim \mathcal{N}\left(Z\alpha + A\beta, \gamma^{2}I_{n}\right), (n\times n) \\
  \alpha|\sigma^{2} &\sim \mathcal{N}\left(0, \frac{\sigma^{2}}{m}I_{2m}\right), (2m\times 1) \\
  \beta &\sim \mathcal{N}\left(0, \Sigma_{\beta}\right), (s\times 1)  \\
  \lambda &\sim \mathcal{N}\left(\mu_{\lambda}, \Sigma_{\lambda}\right), (d \times 1)\\
  \sigma &\sim \opn{half-Cauchy}\left(A_{\sigma}\right)\\
  \gamma &\sim \opn{half-Cauchy}\left(A_{\gamma}\right)
\end{align*}
where $A$ is the design matrix for the random effects and $\beta$ is the parameter vector of the random effects.
\subsection{Lower bound}
$$
  p\left(y, \theta \right) = \mathcal{N}\left(y|Z\alpha + A\beta, \gamma^{2}I_{n}\right) \mathcal{N}\left(\alpha \middle|0, \frac{\sigma^{2}}{m}I_{2m}\right)\mathcal{N}\left(\beta|0,\Sigma_{\beta}\right)\mathcal{N}\left(\lambda,\mu_{\lambda},\Sigma_{\lambda}\right)\operatorname{HC}\left(A_{\sigma}\right)\opn{HC}\left(A_{\gamma}\right)
$$
\begin{align*}
  \textcolor{myorange}{(1)}\;\mathbb{E}\left[\log p\left(y|\theta \right)\right] &= -\frac{n}{2}\log \left(2\pi \right) - \frac{n}{2}\mathbb{E}\left[\log \gamma^{2}\right] \\
  & \,\quad - \frac{1}{2}\mathbb{E}\left[\frac{1}{\gamma^{2}}\right] \left\{y^{T}y -2 \left(\mathbb{E}\left[Z\right]m_{\alpha}+Am_{\beta} \right)^{T}y + \Tr\left(\mathbb{E}\left[Z^{T}Z\right]S_{\alpha} \right) +m_{\alpha}^{T}\mathbb{E}\left[Z^{T}Z\right]m_{\alpha}\right. \\
  &\quad\, + 2m_{\beta}^{T}A^{T}\mathbb{E}\left[Z\right]m_{\alpha} + \left. \Tr\left(A^{T}A S_{\beta}\right) + m_{\beta}^{T}A^{T}A m_{\beta} \right\}\\
  \textcolor{myorange}{(2)}\;\mathbb{E}\left[\log p \left(\alpha|\sigma\right)\right] &= -m \log \left(2\pi \right) -m \mathbb{E}\left[\log \sigma^{2}\right] +m\log m -\frac{m}{2}\mathbb{E}\left[\frac{1}{\sigma^{2}}\right]\Tr \left(S_{\alpha} +m_{\alpha}m_{\alpha}^{T}\right)\\
  \textcolor{myorange}{(3)}\;\mathbb{E}\left[\log p\left(\beta\right)\right] &= -\frac{s}{2}\log \left(2\pi \right) -\frac{1}{2}\log \left|\Sigma_{\beta}\right| -\frac{1}{2}\left\{\Tr \left(\Sigma_{\beta}^{-1}S_{\beta}\right) +m_{\beta}^{T}\Sigma_{\beta}^{-1}m_{\beta} \right\}\\
  \textcolor{myorange}{(4)}\;\mathbb{E}\left[\log p\left(\lambda\right)\right] &= -\frac{d}{2}\log  \left(2\pi \right) -\frac{1}{2}\log \left|\Sigma_{\lambda}\right| -\frac{1}{2}\left(m_{\lambda}-\mu_{\lambda}\right)^{T}\Sigma_{\lambda}^{-1}\left(m_{\lambda}-\mu_{\lambda}\right) - \frac{1}{2}\Tr\left(\Sigma_{\lambda}^{-1}S_{\lambda}\right) \\
  \textcolor{myorange}{(5)}\;\mathbb{E}\left[\log p\left(\sigma\right)\right] &= \log \left(2A_{\sigma}\right) -\log \pi -\mathbb{E}\left[\log \left(A_{\sigma}^{2}+\sigma^{2}\right)\right]\\
  \textcolor{myorange}{(6)}\;\mathbb{E}\left[\log p \left(\gamma\right)\right] &= \log \left(2A_{\gamma}\right) -\log \pi -\mathbb{E}\left[\log \left(A_{\gamma}^{2}+\gamma^{2}\right)\right] \\
  \textcolor{myblue}{(1)}\;\mathbb{E}\left[\log q\left(\alpha\right)\right] &= -m\log \left(2\pi \right) -\frac{1}{2}\log \left|S_{\alpha}\right| -m\\
  \textcolor{myblue}{(2)}\;\mathbb{E}\left[\log q\left(\lambda\right)\right] &= -\frac{d}{2}\log \left(2\pi \right) -\frac{1}{2}\log \left|S_{\lambda}\right| -\frac{d}{2}\\
  \textcolor{myblue}{(3)}\;\mathbb{E}\left[\log q\left(\beta\right)\right] &= -\frac{s}{2}\log \left(2\pi \right) -\frac{1}{2}\log \left|S_{\beta}\right| -\frac{s}{2}\\
  \textcolor{myblue}{(4)}\;\mathbb{E}\left[\log q\left(\sigma\right)\right] &= -C_{\sigma}\frac{\mathcal{H}\left(2m, C_{\sigma}, A_{\sigma}^{2}\right)}{\mathcal{H}\left(2m-2, C_{\sigma}, A_{\sigma}^{2}\right)} -\log \mathcal{H}\left(2m-2, C_{\sigma}, A_{\sigma}^{2}\right) -2m \mathbb{E}\left[\log \sigma\right] -\mathbb{E}\left[\log \left(A_{\sigma}^{2}+\sigma^{2}\right)\right]\\
  \textcolor{myblue}{(5)}\;\mathbb{E}\left[\log q\left(\gamma\right)\right] &= -C_{\gamma}\frac{\mathcal{H}\left(n, C_{\gamma}, A_{\gamma}^{2}\right)}{\mathcal{H}\left(n-2, C_{\gamma}, A_{\gamma}^{2}\right)} -\log \mathcal{H}\left(n-2, C_{\gamma}, A_{\gamma}^{2}\right) -n \mathbb{E}\left[\log \gamma\right] -\mathbb{E}\left[\log \left(A_{\gamma}^{2}+\gamma^{2}\right)\right]
\end{align*}
$$
  \mathcal{L} = \textcolor{myorange}{(1)} + \textcolor{myorange}{(2)} + \textcolor{myorange}{(3)} + \textcolor{myorange}{(4)} + \textcolor{myorange}{(5)} + \textcolor{myorange}{(6)} - \textcolor{myblue}{(1)} - \textcolor{myblue}{(2)} - \textcolor{myblue}{(3)} - \textcolor{myblue}{(4)} - \textcolor{myblue}{(5)}
$$
\section{GP Logistic Model}
\subsection{Model specifications}
For logistic models, we first postulate a link function, $g\left(\cdot \right)$ for the predictors.
\begin{align*}
  y &= g^{-1}\left(\eta\right) + \epsilon, \quad \epsilon \sim \mathcal{N}\left(0, \gamma^{2}I_{n}\right)\\
  g\left(\mathbb{E}\left[y\right]\right) &= \eta\\
  \eta &= f\left(x\right) + A\beta\\
  g^{-1}\left(x\right) &= e^{x}/\left(1+e^{x}\right)\\
  f &\sim \mathcal{GP}\left(m\left(\cdot \right), \kappa\left(\cdot, \cdot \right)\right)
\end{align*}
Since $y$ is a Bernoulli random variable, $\mathbb{E}\left[y\right] = \mathbb{P}\left(y=1\right)$. Without loss of generality, we will assume the mean function to be zero and that the Gaussian process has a sparse approximation representation as in Tan \& Nott. Therefore,
$$
  f\left(x\right) \approx \sum_{r=1}^{m}\left[a_{r}\cos \left\{\left(s_{r} \odot x\right)^{T}\lambda \right\}+b_{r}\sin \left\{\left(s_{r}\odot x\right)^{T}\lambda \right\}\right]
$$
and this could further be represented in matrix form which reduces this to a linear model.
$$
  f\left(x\right) = Z\alpha
$$
\begin{align*}
  \eta &= Z\alpha + A\beta\\
  y &= \exp\left\{\left(Z\alpha+A\beta\right) -\log \left(\bs{1}+\exp \left\{Z\alpha +A\beta \right\}\right) \right\} + \epsilon
\end{align*}
Every scalar function applied to a vector or a matrix is done so elementwise. We think of the following priors:
\begin{align*}
  \alpha|\sigma &\sim \mathcal{N}\left(0, \frac{\sigma^{2}}{m}I_{2m}\right)\\
  \beta &\sim \mathcal{N}\left(\mu_{\beta}, \Sigma_{\beta}\right) \\
  \lambda &\sim \mathcal{N}\left(\mu_{\lambda}, \Sigma_{\lambda}\right) \\
  \sigma &\sim \opn{half-Cauchy}\left(A_{\sigma}\right) \\
  \gamma &\sim \opn{half-Cauchy}\left(A_{\gamma}\right)\\
  \theta &= \left(\alpha, \beta, \lambda, \sigma, \gamma\right)
\end{align*}
\begin{align*}
  \log p\left(y, \theta \right) &= y^{T}\left(Z\alpha+A\beta\right) -\bs{1}_{n}^{T}\log \left(\bs{1}_{n}+\exp \left\{Z\alpha + A\beta \right\}\right) -\left(m+\frac{s+d}{2}\right)\log \left(2\pi\right) -m\log \sigma^{2} + m\log m \\
  &\quad -\frac{m}{2\sigma^{2}}\alpha^{T}\alpha -\frac{1}{2}\log \left|\Sigma_{\beta}\right| -\frac{1}{2}\left(\beta-\mu_{\beta}\right)^{T}\Sigma_{\beta}^{-1}\left(\beta-\mu_{\beta}\right) -\frac{1}{2}\log \left|\Sigma_{\lambda}\right| -\frac{1}{2}\left(\lambda-\mu_{\lambda}\right)^{T}\Sigma_{\lambda}^{-1}\left(\lambda-\mu_{\lambda}\right) \\
  &\quad + \log \left(2A_{\sigma}\right)+\log \left(2A_{\gamma}\right) 2\log \pi -\log \left(A_{\sigma}^{2}+\sigma^{2}\right) -\log \left(A_{\gamma}^{2}+\gamma^{2}\right)
\end{align*}
Because $-\bs{1}_{n}^{T}\log \left(\bs{1}_{n}+\exp \left\{Z\alpha + A\beta \right\}\right)$ is analytically intractable for expectation which is essentially integration, we come up with the following approximation:
\begin{align*}
  -\log \left(1+e^{x}\right) &= \max_{\xi \in \RR} \left\{B\left(\xi\right)x^{2} -\frac{1}{2}x+C\left(\xi \right) \right\}, \quad \forall x\in\RR\\
  B\left(\xi\right) &= -\opn{tanh}\left(\xi/2\right)/\left(4\xi\right)\\
  C\left(\xi\right) &= \xi/2 -\log \left(1+e^{\xi}\right) +\xi \opn{tanh}\left(\xi/2\right)/4
\end{align*}
then
\begin{align*}
  -\bs{1}_{n}^{T}\log \left\{\bs{1}_{n}^{T}+\exp \left(Z\alpha+A\beta\right) \right\} &\ge \bs{1}_{n}^{T}\left\{B\left(\xi\right) \odot \left(Z\alpha+A\beta\right)^{2} -\frac{1}{2}\left(Z\alpha+A\beta\right) +C\left(\xi\right) \right\}\\
  &= \left(Z\alpha+A\beta\right)^{T}\opn{Dg}\left\{B\left(\xi\right)\right\}\left(Z\alpha+A\beta\right) -\frac{1}{2}\bs{1}_{n}^{T}\left(Z\alpha+A\beta\right) +\bs{1}_{n}^{T}C\left(\xi\right),
\end{align*}
where $\xi = \left(\xi_{1}, \ldots , \xi_{n}\right)$.
\begin{align*}
  \log \underline{p}\left(y, \theta;\xi \right) &= y^{T}\left(Z\alpha+A\beta\right)+ \left(Z\alpha+A\beta\right)^{T}\opn{Dg}\left\{B\left(\xi\right)\right\}\left(Z\alpha+A\beta\right) -\frac{1}{2}\bs{1}_{n}^{T}\left(Z\alpha+A\beta\right) +\bs{1}_{n}^{T}C\left(\xi\right) \\
  &\quad-\left(m+\frac{s+d}{2}\right)\log \left(2\pi\right) -m\log \sigma^{2} + m\log m -\frac{m}{2\sigma^{2}}\alpha^{T}\alpha -\frac{1}{2}\log \left|\Sigma_{\beta}\right| -\frac{1}{2}\left(\beta-\mu_{\beta}\right)^{T}\Sigma_{\beta}^{-1}\left(\beta-\mu_{\beta}\right)\\
  &\quad -\frac{1}{2}\log \left|\Sigma_{\lambda}\right| -\frac{1}{2}\left(\lambda-\mu_{\lambda}\right)^{T}\Sigma_{\lambda}^{-1}\left(\lambda-\mu_{\lambda}\right) + \log \left(2A_{\sigma}\right)+\log \left(2A_{\gamma}\right) 2\log \pi -\log \left(A_{\sigma}^{2}+\sigma^{2}\right)\\
  &\quad-\log \left(A_{\gamma}^{2}+\gamma^{2}\right)
\end{align*}
\section{Sparse GP Probit Regression}
\subsection{Bayesian probit regression without random effects}
The ordinary formulation goes like this:
$$
  \mathbb{P}\left(y_{i}=1|x_{i}, \beta\right) = \Phi\left(x_{i}^{T}\beta\right),
$$
where $\Phi$ is the cumulative distribution of standard Gaussian distribution. Since the following likelihood
$$
  \mathbb{P}\left(Y=y|X, \beta\right) = \prod_{i=1}^{n}\Phi\left(x_{i}^{T}\beta\right)^{y_{i}}\left[1-\Phi\left(x_{i}^{T}\beta\right)\right]^{1-y_{i}}
$$
hinders the tractable calculation of the posterior, we devise the following latent variable:
$$
  Z_{i} = x_{i}^{T}\beta + \epsilon_{i}, \quad \epsilon_{i} \sim \mathcal{N}\left(0, 1\right)
$$
for $i = 1, \ldots , n$ and let
\begin{align*}
  y_{i} = \begin{cases}1, & \text{if $Z_{i} \ge 0$}\\ 0, & \text{otherwise} \end{cases}.
\end{align*}
It automatically follows that
\begin{align*}
  Z_{i}|y_{i}=0, x_{i}, \beta &\sim \mathcal{N}\left(x_{i}^{T}\beta, 1\right) \bs{1}\left[Z_{i} < 0\right]\\
  Z_{i}|y_{i}=1, x_{i}, \beta &\sim \mathcal{N}\left(x_{i}^{T}\beta, 1\right) \bs{1}\left[Z_{i}\ge 0\right]
\end{align*}
suggesting a truncated Gaussian for each $Z_{i}$ depending on what value $y_{i}$ takes on. According to probit regression without random effects, the introduction of latent variables enables the tractable computation of the posterior with regards to the parameters.
\begin{align*}
  \pi\left(Z, y, \beta\right) &= C\pi\left(\beta\right)\prod_{i=1}^{n}\left\{1\left[Z_{i}\ge 0\right]1\left[y_{i}=1\right] + 1\left[Z_{i}<0\right]1\left[y_{i}=0\right] \right\}\phi\left(Z_{i} - x_{i}^{T}\beta\right)
\end{align*}
where $C$ is the proportionality constant and $\phi$ is the density of standard Gaussian. Following the mean-field approximation and computing the optimal density for $Z_{i}$, it becomes a truncated normal about zero with mean $x_{i}^{T}\beta$ and variance $1$.
\subsection{Incorporating random effects and nonparametric statistics}
Taking one step forward, the random effects can be taken into consideration in many situations. Furthermore, we often do not know exactly what functional form the parameters should take on. Such situations greatly call for the use of nonparametric statistics.
\begin{align*}
  \mathbb{P}\left(y_{i}=1|f\right) = \Phi\left(f\left(x_{i}\right)\right).
\end{align*}
In the previous section where linear probit regression was considered, the functional form was assumed to be linear, i.e. $x_{i}^{T}\beta$. However, we now insist that $f$ could be any function. Adopting Gaussion process prior for $f$ and using sparse approximation,
\begin{align*}
  \mathbb{P}\left(y_{i}=1|f\right) = \Phi\left(Z_{i}^{T}\alpha\right).
\end{align*}
Note that $Z_{i}$ is different from the one we used in the previous section. Moreover, we will assume there exist additive random effects:
$$
  \mathbb{P}\left(y_{i}=1|f, \alpha, \beta\right) = \Phi\left(z_{i}^{T}\alpha + a_{i}^{T}\beta\right).
$$
Assume latent variables
$$
  y_{i}^{*} = z_{i}^{T}\alpha + a_{i}^{T}\beta + \epsilon, \quad \epsilon \sim \mathcal{N}\left(0, 1\right).
$$
Let 
\begin{align*}
  y_{i} = \begin{cases}
    1, & \text{if $y_{i}^{*} \ge 0$}\\
    0, & \text{if $y_{i}^{*} < 0$}.
  \end{cases}
\end{align*}
Let's assume priors
\begin{align*}
  \alpha|\sigma &\sim \mathcal{N}\left(0, \frac{\sigma^{2}}{m}I_{2m}\right)\\
    \beta &\sim \mathcal{N}\left(\mu_{\beta}, \Sigma_{\beta}\right) \\
    \lambda &\sim \mathcal{N}\left(\mu_{\lambda}, \Sigma_{\lambda}\right) \\
    \sigma &\sim \opn{half-Cauchy}\left(A_{\sigma}\right)\\
    \theta &= \left(\alpha, \beta, \lambda, \sigma \right)
\end{align*}
\begin{align*}
  \pi\left(y, y^{*}, \theta \right) &= C\pi\left(\alpha|\sigma\right)\pi\left(\beta\right)\pi\left(\lambda\right)\pi\left(\sigma\right)\prod_{i=1}^{n}\left\{1\left[y_{i}^{*}\ge 0\right]1\left[y_{i}=1\right] + 1\left[y_{i}^{*}<0\right]1\left[y_{i}=0\right] \right\}\phi\left(y_{i}^{*}-z_{i}^{T}\alpha_{i} - a_{i}^{T}\beta\right)
\end{align*}
\subsubsection{$q\left(\alpha\right)$}
Let $y^{*} = \begin{bmatrix}y_{1}^{*} & \ldots & y_{n}^{*}\end{bmatrix}^{T}$.
\begin{align*}
  \log q\left(\alpha\right) &\propto \left\langle \log \pi\left(\alpha|\sigma\right)\prod_{i=1}^{n}\phi\left(y_{i}^{*}-z_{i}^{T}\alpha - a_{i}^{T}\beta\right)\right\rangle\\
  &\propto -\frac{m}{2}\left\langle \frac{1}{\sigma^{2}}\right\rangle\alpha^{T}\alpha +\left\langle \sum_{i=1}^{n}\left[-\frac{1}{2}\left\{{y_{i}^{*}}^{2} -2\left(z_{i}^{T}\alpha + a_{i}^{T}\beta\right)y_{i}^{*} + \left(z_{i}^{T}\alpha + a_{i}^{T}\beta\right)^{2}\right\}\right]\right\rangle\\
  &\propto -\frac{m}{2}\left\langle \frac{1}{\sigma^{2}} \right\rangle \alpha^{T}\alpha + \left\langle -\frac{1}{2}\sum_{i=1}^{n}\left[-2y_{i}^{*}z_{i}^{T}\alpha +\alpha^{T}z_{i}z_{i}^{T}\alpha +2\beta^{T}a_{i}z_{i}^{T}\alpha \right]\right\rangle\\
  &\propto -\frac{m}{2}\left\langle \frac{1}{\sigma^{2}}\right\rangle \alpha^{T}\alpha -\frac{1}{2}\left\langle -2{y^{*}}^{T}Z\alpha + \alpha^{T}\left(\sum_{i=1}^{n}z_{i}z_{i}^{T}\right)\alpha + 2\beta^{T}\left(\sum_{i=1}^{n}a_{i}z_{i}^{T}\right)\alpha\right\rangle \\
  &\propto -\frac{m}{2}\left\langle \frac{1}{\sigma^{2}}\right\rangle \alpha^{T}\alpha -\frac{1}{2}\left\{-2\left\langle {y^{*}}^{T}\right\rangle \left\langle Z\right\rangle \alpha + \alpha^{T}\left\langle Z^{T}Z \right\rangle \alpha + 2\left\langle\beta\right\rangle^{T} A^{T}\left\langle Z\right\rangle \alpha \right\}\\
  &\propto -\frac{1}{2}\left\{\alpha^{T}\left(m\left\langle \frac{1}{\sigma^{2}}\right\rangle I_{2m} + \left\langle Z^{T}Z \right\rangle\right)\alpha-2\left(\left\langle Z\right\rangle^{T}\left\langle y^{*}\right\rangle -\left\langle Z \right\rangle^{T}A^{T}\left\langle \beta\right\rangle \right)^{T}\alpha \right\}\\
  q\left(\alpha\right) &= \mathcal{N}\left(\mu_{q\left(\alpha\right)}, \Sigma_{q\left(\alpha\right)}\right)\\
  \mu_{q\left(\alpha\right)} &= \Sigma_{q\left(\alpha\right)}\left(\left\langle Z\right\rangle^{T}\left\langle y^{*}\right\rangle -\left\langle Z \right\rangle^{T}A^{T}\left\langle \beta\right\rangle \right)\\
  \Sigma_{q\left(\alpha\right)} &= \left(m\left\langle \frac{1}{\sigma^{2}}\right\rangle I_{2m} + \left\langle Z^{T}Z \right\rangle\right)^{-1}
\end{align*}
\subsubsection{$q\left(\beta\right)$}
\begin{align*}
  \log q\left(\beta\right) &\propto \log \pi\left(\beta\right)\prod_{i=1}^{n}\phi\left(y_{i}^{*}-z_{i}^{T}\alpha - a_{i}^{T}\beta \right) \\
  &\propto -\frac{s}{2}\log \left(2\pi\right) -\frac{1}{2}\log \left|\Sigma_{\beta}\right| -\frac{1}{2}\left(\beta-\mu_{\beta}\right)^{T}\Sigma_{\beta}^{T}\left(\beta-\mu_{\beta}\right) \\
  &\quad +\left\langle\sum_{i=1}^{T}\left[-\frac{1}{2}\left(y_{i}^{*} -2\left(z_{i}^{T}\alpha + a_{i}^{T}\beta\right)y_{i}^{*} + \left(z_{i}^{T}\alpha + a_{i}^{T}\beta\right)^{2}\right)\right]\right\rangle\\
  &\propto -\frac{1}{2}\left(\beta^{T}\Sigma_{\beta}^{-1}\beta -2\mu_{\beta}^{T}\Sigma_{\beta}^{-1}\beta\right) +\left\langle\sum_{i=1}^{n}\left[-\frac{1}{2}\left(-2y_{i}^{*}a_{i}^{T}\beta + 2\alpha^{T}z_{i}a_{i}^{T}\beta + \beta^{T}a_{i}a_{i}^{T}\beta\right)\right]\right\rangle\\
  &\propto -\frac{1}{2}\left(\beta^{T}\Sigma_{\beta}^{-1}\beta-2\mu_{\beta}^{T}\Sigma_{\beta}^{-1}\beta\right) - \frac{1}{2}\left(-2\left\langle{y^{*}}^{T}\right\rangle A\beta + 2\left\langle\alpha^{T}\right\rangle \left\langle Z^{T}\right\rangle A\beta + \beta^{T}A^TA\beta\right)\\
  &\propto -\frac{1}{2}\left\{\beta^{T}\Sigma_{\beta}^{-1}\beta + \beta^{T}A^{T}A\beta -2\mu_{\beta}^{T}\Sigma_{\beta}^{-1}\beta +2\left\langle \alpha^{T}\right\rangle \left\langle Z^{T}\right\rangle A\beta -2\left\langle{y^{*}}^{T}\right\rangle A\beta\right\}\\
  &\propto -\frac{1}{2}\left\{\beta^{T}\left(\Sigma_{\beta}^{-1} + A^{T}A\right)\beta -2\left(\Sigma_{\beta}^{-1}\mu_{\beta} +A^{T} \left\langle y^{*}\right\rangle -A^{T}\left\langle Z\right\rangle \left\langle \alpha\right\rangle \right)^{T}\beta\right\}\\
  q\left(\beta\right) &= \mathcal{N}\left(\mu_{q\left(\beta\right)}, \Sigma_{q\left(\beta\right)}\right)\\
  \mu_{q\left(\beta\right)} &= \Sigma_{q\left(\beta\right)} \left(\Sigma_{\beta}^{-1}\mu_{\beta} +A^{T} \left\langle y^{*}\right\rangle -A^{T}\left\langle Z\right\rangle \left\langle \alpha\right\rangle \right)\\
  \Sigma_{q\left(\beta\right)} &= \left(\Sigma_{\beta}^{-1} + A^{T}A\right)^{-1}
\end{align*}
\subsubsection{$q\left(y^{*}\right)$}
\begin{align*}
  \log q\left(y^{*}\right) &\propto \left\langle \log \prod_{i=1}^{n}\left\{1\left[y_{i}^{*}\ge 0\right]1\left[y_{i}=1\right] + 1\left[y_{i}^{*}<0\right]1\left[y_{i}=0\right] \right\}\phi\left(y_{i}^{*}-z_{i}^{T}\alpha -a_{i}^{T}\beta\right) \right\rangle \\
  &\propto \sum_{i=1}^{n}\log\left\{1\left[y_{i}^{*}\ge 0\right]1\left[y_{i}=1\right] + 1\left[y_{i}^{*}<0\right]1\left[y_{i}=0\right] \right\} \\
  &\quad + \left\langle\sum_{i=1}^{n}\left[-\frac{1}{2}\left({y_{i}^{*}}^{2} -2\left(z_{i}^{T}\alpha +a_{i}^{T}\beta\right)y_{i}^{*} +\left(z_{i}^{T}\alpha +a_{i}^{T}\beta\right)^{2}\right) \right] \right\rangle\\
  &\propto \sum_{i=1}^{n}\log\left\{1\left[y_{i}^{*}\ge 0\right]1\left[y_{i}=1\right] + 1\left[y_{i}^{*}<0\right]1\left[y_{i}=0\right] \right\} \\
  &\quad -\frac{1}{2}\left({y^{*}}^{T}y^{*} -2\left(\left\langle Z\right\rangle \left\langle \alpha\right\rangle +A\left\langle\beta\right\rangle \right)^{T}y^{*}\right)\\
  q\left(y^{*}\right) &= \mathcal{TN}\left(\left\langle Z\right\rangle \left\langle \alpha\right\rangle +A\left\langle\beta\right\rangle, I_{n}\right)
\end{align*}
where $\mathcal{TN}$ indicates truncated normal distribution, in this case multivariate. Each element of $y^{*}$ is truncated at zero. The direction of truncation remains the same. Assuming $y_{i}=1$,
\begin{align*}
  \mu_{q\left(y_{i}^{*}\right)} &= \left\langle z_{i}^{T}\right\rangle \mu_{q\left(\alpha\right)} + a_{i}^{T}\mu_{q\left(\beta\right)} + \frac{\phi\left(\left\langle z_{i}^{T}\right\rangle \mu_{q\left(\alpha\right)} + a_{i}^{T}\mu_{q\left(\beta\right)}\right)}{\Phi\left(\left\langle z_{i}^{T}\right\rangle \mu_{q\left(\alpha\right)} + a_{i}^{T}\mu_{q\left(\beta\right)}\right)}
\end{align*}
\subsubsection{$q\left(\sigma\right)$}
\begin{align*}
  \log q\left(\sigma\right) &\propto \left\langle \log \pi\left(\alpha|\sigma\right)\pi\left(\sigma\right) \right\rangle\\
  &\propto -m\log \sigma^{2} -\frac{m}{2}\left\langle \alpha^{T}\alpha \right\rangle /\sigma^{2} -\log \left(A_{\sigma}^{2} +\sigma^{2}\right)\\
  q\left(\sigma\right) &\propto \frac{\exp \left(-C_{\sigma}/\sigma^{2}\right)}{\sigma^{2m}\left(A_{\sigma}^{2}+\sigma^{2}\right)}\\
  C_{\sigma} &= \frac{m}{2}\left(\Tr\left(\Sigma_{q\left(\alpha\right)}\right) + \mu_{q\left(\alpha\right)}^{T}\mu_{q\left(\alpha\right)}\right)
\end{align*}
\subsubsection{LB: $\mathbb{E}\left[\log p\left(y, y^{*}|\theta\right) \right]$}
Recall that 
\begin{align*}
  \pi\left(y, y^{*}, \theta \right) &= C\pi\left(\alpha|\sigma\right)\pi\left(\beta\right)\pi\left(\lambda\right)\pi\left(\sigma\right)\prod_{i=1}^{n}\left\{1\left[y_{i}^{*}\ge 0\right]1\left[y_{i}=1\right] + 1\left[y_{i}^{*}<0\right]1\left[y_{i}=0\right] \right\}\phi\left(y_{i}^{*}-z_{i}^{T}\alpha_{i} - a_{i}^{T}\beta\right)\\
  \mathcal{L} &= \mathbb{E}\left[\log \pi\left(y, y^{*}, \theta \right) \right] - \mathbb{E}\left[\log q\left(y^{*}\right)q\left(\theta\right) \right]\\
  &= \mathbb{E}\left[\log p\left(y|y^{*}, \theta \right) \right] + \mathbb{E}\left[\log p\left(y^{*}|\theta \right) \right] + \mathbb{E}\left[\log p \left(\theta \right) \right] - \mathbb{E}\left[\log q\left(\theta\right) \right]
\end{align*}
The above equation includes all the nodes in the graphical model. Therefore, we compute the expecation of the logarithm of each node.
\begin{align*}
  \log p\left(y, y^{*}|\theta\right) &= \log \prod_{i=1}^{n}\left\{1\left[y_{i}^{*} \ge 0\right]1\left[y_{i}=1\right] + 1\left[y_{i}^{*}<0\right]1\left[y_{i}=0\right] \right\}\phi\left(y_{i}^{*} - z_{i}^{T}\alpha -a_{i}^{T}\beta\right)\\
  &\implies \mathbb{E}\left[\sum_{i=1}^{n}\log\left\{1\left[y_{i}^{*} \ge 0\right]1\left[y_{i}=1\right] + 1\left[y_{i}^{*}<0\right]1\left[y_{i}=0\right] \right\}  \right] + \sum_{i=1}^{n} \mathbb{E}\left[\log \phi\left(y_{i}^{*} - z_{i}^{T}\alpha - a_{i}^{T}\beta\right) \right]
\end{align*}
If we set $U = \left\{1\left[y_{i}^{*} \ge 0\right]1\left[y_{i}=1\right] + 1\left[y_{i}^{*}<0\right]1\left[y_{i}=0\right] \right\}$ and let it be a new random variable, it is always $1$ with probability 1. Therefore, the logarithm of $U$, i.e. $\log U$ is always $0$ which terminates the need for calculating the expected value since it's $0$ regardless. We proceed to the remaining terms.
\begin{align*}
  \sum_{i=1}^{n}\mathbb{E}\left[\log \phi\left(y_{i}^{*} - z_{i}^{T}\alpha - a_{i}^{T}\beta\right) \right] \qquad \cdots (1)
\end{align*}
The entropy of the variational distribution of $q\left(y_{i}^{*}\right)$ should be coupled with eqn $(1)$ to simplify the calculation.
\begin{align*}
  &\sum_{i=1}^{n}\mathbb{E}\left[\log \phi\left(y_{i}^{*}-z_{i}^{T}\alpha - a_{i}^{T}\beta\right) - \log \phi\left(y_{i}^{*}-\left\langle z_{i}^{T}\right\rangle \mu_{q\left(\alpha\right)} - a_{i}^{T}\mu_{q\left(\beta\right)}\right)\right] \\
  &+\sum_{i=1}^{n} \log \left(\left\{\Phi\left(\left\langle z_{i} \right\rangle^{T}\mu_{q\left(\alpha\right)} + a_{i}^{T}\mu_{q\left(\beta\right)}\right) \right\}^{y_{i}}\left\{1-\Phi\left(\left\langle z_{i} \right\rangle^{T}\mu_{q\left(\alpha\right)} + a_{i}^{T}\mu_{q\left(\beta\right)}\right) \right\}^{1-y_{i}} \right)\\
  &= \sum_{i=1}^{n}\mathbb{E}\left[-\frac{1}{2}\left(- 2\left(z_{i}^{T}\alpha + a_{i}^{T}\beta \right)y_{i}^{*} + \left(z_{i}^{T}\alpha + a_{i}^{T}\beta \right)^{2} +2\left(\left\langle z_{i}\right\rangle^{T}\mu_{q\left(\alpha\right)}+a_{i}^{T}\mu_{q\left(\beta\right)} \right)y_{i}^{*}\right)\right]\\
  &\quad +\frac{1}{2}\sum_{i=1}^{n}\left( \left(\left\langle z_{i}\right\rangle^{T}\mu_{q\left(\alpha\right)} +a_{i}^{T}\mu_{q\left(\beta\right)} \right)^{2}\right)\\
  &\quad +\sum_{i=1}^{n} \log \left(\left\{\Phi\left(\left\langle z_{i} \right\rangle^{T}\mu_{q\left(\alpha\right)} + a_{i}^{T}\mu_{q\left(\beta\right)}\right) \right\}^{y_{i}}\left\{1-\Phi\left(\left\langle z_{i} \right\rangle^{T}\mu_{q\left(\alpha\right)} + a_{i}^{T}\mu_{q\left(\beta\right)}\right) \right\}^{1-y_{i}} \right)\\
  \end{align*}
  \begin{empheq}[box=\fbox]{align*}
    &= \textcolor{myorange}{-\frac{1}{2}\left(\Tr \left(\left\langle Z^{T}Z \right\rangle \Sigma_{q\left(\alpha\right)}\right) + \mu_{q\left(\alpha\right)}^{T}\left(\left\langle Z^{T}Z\right\rangle - \left\langle Z \right\rangle^{T}\left\langle Z \right\rangle \right)\mu_{q\left(\alpha\right)} + \Tr \left(A^{T}A\Sigma_{q\left(\beta\right)}\right) \right)}\\
    &\quad \textcolor{myorange}{+\sum_{i=1}^{n} \log \left(\left\{\Phi\left(\left\langle z_{i} \right\rangle^{T}\mu_{q\left(\alpha\right)} + a_{i}^{T}\mu_{q\left(\beta\right)}\right) \right\}^{y_{i}}\left\{1-\Phi\left(\left\langle z_{i} \right\rangle^{T}\mu_{q\left(\alpha\right)} + a_{i}^{T}\mu_{q\left(\beta\right)}\right) \right\}^{1-y_{i}} \right)}
\end{empheq}
\end{document}