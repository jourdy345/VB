\documentclass[11pt]{article}

% This first part of the file is called the PREAMBLE. It includes
% customizations and command definitions. The preamble is everything
% between \documentclass and \begin{document}.

\usepackage[margin=0.6in]{geometry} % set the margins to 1in on all sides
\usepackage{graphicx} % to include figures
\usepackage{amsmath} % great math stuff
\usepackage{amsfonts} % for blackboard bold, etc
\usepackage{amsthm} % better theorem environments
% various theorems, numbered by section
\usepackage{amssymb}
\usepackage[utf8]{inputenc}
\usepackage{booktabs}
\usepackage{array}
\usepackage{courier}
\usepackage[usenames, dvipsnames]{color}
\usepackage{titlesec}
\usepackage{empheq}
\usepackage{tikz}

\newcommand\encircle[1]{%
  \tikz[baseline=(X.base)] 
    \node (X) [draw, shape=circle, inner sep=0] {\strut #1};}
 
% Command "alignedbox{}{}" for a box within an align environment
% Source: http://www.latex-community.org/forum/viewtopic.php?f=46&t=8144
\newlength\dlf  % Define a new measure, dlf
\newcommand\alignedbox[2]{
% Argument #1 = before & if there were no box (lhs)
% Argument #2 = after & if there were no box (rhs)
&  % Alignment sign of the line
{
\settowidth\dlf{$\displaystyle #1$}  
    % The width of \dlf is the width of the lhs, with a displaystyle font
\addtolength\dlf{\fboxsep+\fboxrule}  
    % Add to it the distance to the box, and the width of the line of the box
\hspace{-\dlf}  
    % Move everything dlf units to the left, so that & #1 #2 is aligned under #1 & #2
\boxed{#1 #2}
    % Put a box around lhs and rhs
}
}


\newtheorem{thm}{Theorem}[section]
\newtheorem{lem}[thm]{Lemma}
\newtheorem{prop}[thm]{Proposition}
\newtheorem{cor}[thm]{Corollary}
\newtheorem{conj}[thm]{Conjecture}

\setcounter{secnumdepth}{4}

\titleformat{\paragraph}
{\normalfont\normalsize\bfseries}{\theparagraph}{1em}{}
\titlespacing*{\paragraph}
{0pt}{3.25ex plus 1ex minus .2ex}{1.5ex plus .2ex}

\definecolor{myblue}{RGB}{72, 165, 226}
\definecolor{myorange}{RGB}{222, 141, 8}

\setlength{\heavyrulewidth}{1.5pt}
\setlength{\abovetopsep}{4pt}


\DeclareMathOperator{\id}{id}
\DeclareMathOperator{\argmin}{\arg\!\min}
\DeclareMathOperator{\Tr}{Tr}

\newcommand{\bd}[1]{\mathbf{#1}} % for bolding symbols
\newcommand{\RR}{\mathbb{R}} % for Real numbers
\newcommand{\ZZ}{\mathbb{Z}} % for Integers
\newcommand{\col}[1]{\left[\begin{matrix} #1 \end{matrix} \right]}
\newcommand{\comb}[2]{\binom{#1^2 + #2^2}{#1+#2}}
\newcommand{\bs}{\boldsymbol}
\newcommand{\opn}{\operatorname}
\begin{document}
\nocite{*}

\title{Cosine Basis Logistic}

\author{Daeyoung Lim\thanks{Prof. Taeryon Choi} \\
Department of Statistics \\
Korea University}

\maketitle
\section{Logistic}
\subsection{Model}
  \begin{align*}
    y &\sim \opn{Ber}\left(\opn{logit}^{-1}\left(\varphi_{J}\theta_{J}\right)\right)\\
    \theta_{j}| \tau, \psi &\sim \mathcal{N}\left(0, \tau^{2}\exp\left[-j\left|\psi\right|\right]\right)\\
    \tau^{2} &\sim \opn{IG}\left(\frac{r_{0,\tau}}{2}, \frac{s_{0,\tau}}{2}\right)\\
    \psi &\sim \opn{DE}\left(0,\omega_{0}\right)
  \end{align*}
\subsection{Likelihood}
  \begin{align*}
    \ln p\left(y, \Theta\right) &= y^{\top}\varphi_{J}\theta_{J} - 1_{n}^{\top}\ln \left(1_{n} + \exp\left(\varphi_{j}\theta_{J}\right)\right)\\
    &\quad -\frac{1}{2}\left\{\sum_{j=1}^{J}\ln\left(2\pi\tau^{2}\right) - j\left|\psi\right| + \frac{\theta_{j}^{2}e^{j\left|\psi\right|}}{\tau^{2}}\right\}\\
    &\quad +\frac{r_{0,\tau}}{2}\ln\left(\frac{s_{0,\tau}}{2}\right) - \ln\Gamma\left(\frac{r_{0,\tau}}{2}\right) + \left(\frac{r_{0,\tau}}{2}+1\right)\ln\frac{1}{\tau^{2}} -\frac{s_{0,\tau}}{2}\frac{1}{\tau^{2}}\\
    &\quad -\ln\frac{\omega_{0}}{2} -\omega_{0}\left|\psi\right|
  \end{align*}
\subsection{Getting around the intractability}
\begin{align*}
  -\ln\left(1+e^{x}\right) &= \max_{\xi \in \mathbb{R}}\left\{\lambda\left(\xi\right)x^{2} - \frac{1}{2}x+\Psi\left(\xi\right) \right\}, \quad \forall x \in \mathbb{R}\\
  \lambda\left(\xi\right) &= -\tanh\left(\xi/2\right)/\left(4\xi\right)\\
  \Psi\left(\xi\right) &= \xi/2 - \ln\left(1+e^{\xi}\right) + \xi\tanh\left(\xi/2\right)/4
\end{align*}
then
\begin{align*}
  -1_{n}^{\top}\ln\left(1_{n}+\exp\left(\varphi_{J}\theta_{J}\right)\right) &\geq 1_{n}^{\top}\left\{\lambda\left(\xi\right) \odot \left(\varphi_{J}\theta_{J}\right)^{2} -\frac{1}{2}\varphi_{J}\theta_{J} + \Psi\left(\xi\right) \right\}\\
  &= \theta_{J}^{\top}\varphi_{J}^{\top}\opn{Dg}\left\{\lambda\left(\xi\right) \right\}\varphi_{J}\theta_{J} -\frac{1}{2}1_{n}^{\top}\varphi_{J}\theta_{J} + 1_{n}^{\top}\Psi\left(\xi\right)
\end{align*}
\begin{align*}
  \ln \underline{p}\left(y,\Theta;\xi\right) &= y^{\top}\varphi_{J}\theta_{J} -\frac{1}{2}1_{n}^{\top}\varphi_{J}\theta_{J} +\theta_{J}^{\top}\varphi_{J}^{\top}\opn{Dg}\left\{\lambda\left(\xi\right) \right\}\varphi_{J}\theta_{J} +1_{n}^{\top}\Psi\left(\xi\right) \\
    &\quad -\frac{1}{2}\left\{\sum_{j=1}^{J}\ln\left(2\pi\tau^{2}\right) - j\left|\psi\right| + \frac{\theta_{j}^{2}e^{j\left|\psi\right|}}{\tau^{2}}\right\}\\
    &\quad +\frac{r_{0,\tau}}{2}\ln\left(\frac{s_{0,\tau}}{2}\right) - \ln\Gamma\left(\frac{r_{0,\tau}}{2}\right) + \left(\frac{r_{0,\tau}}{2}+1\right)\ln\frac{1}{\tau^{2}} -\frac{s_{0,\tau}}{2}\frac{1}{\tau^{2}}\\
    &\quad -\ln\frac{\omega_{0}}{2} -\omega_{0}\left|\psi\right|
\end{align*}
\subsection{Update}
\subsubsection{$\theta_{J}$}
\begin{align*}
  \Sigma_{\theta;\xi}^{q} &= \left(-2\varphi_{J}^{\top}\opn{Dg}\left\{\lambda\left(\xi\right) \right\}\varphi_{J}+\frac{r_{q,\tau}}{s_{q,\tau}}\opn{Dg}\left(Q_{1:J}\right)\right)^{-1}\\
  \mu_{\theta;\xi}^{q} &= \Sigma_{\theta;\xi}^{q}\varphi_{J}^{\top}\left(y -\frac{1}{2}1_{n}\right)
\end{align*}
\subsubsection{$\tau^{2}$}
\begin{align*}
  r_{q,\tau} &= r_{0,\tau} + J\\
  s_{q,\tau} &= s_{0,\tau} + \sum_{j=1}^{J}\left(\Sigma_{\theta;\xi,jj}^{q} + {\mu_{\theta;\xi,jj}^{q}}^{2}\right)Q_{j}\left(\mu_{\psi}^{q},{\sigma_{\psi}^{q}}^{2}\right)
\end{align*}
\subsubsection{$\psi$}
\begin{align*}
  {\sigma_{\psi}^{q}}^{2} &= -\frac{1}{2}\left\{\frac{\partial S_{1}}{\partial {\sigma_{\psi}^{q}}^{2}} +\frac{\partial S_{2}}{\partial {\sigma_{\psi}^{q}}^{2}}  \right\}^{-1}\\
  \mu_{\psi}^{q} &= \mu_{\psi}^{q} + {\sigma_{\psi}^{q}}^{2}\left\{\frac{\partial S_{1}}{\partial \mu_{\psi}^{q}} + \frac{\partial S_{2}}{\partial \mu_{\psi}^{q}}\right\}
\end{align*}
\subsubsection{$\xi$}
\begin{align*}
  \xi^{\text{new}} &= \sqrt{\opn{dg}\left\{\varphi_{J}\left(\Sigma_{\theta;\xi}^{q}+\mu_{\theta;\xi}^{q}{\mu_{\theta;\xi}^{q}}^{\top}\right)\varphi_{J}^{\top} \right\}}
\end{align*}
where $\opn{dg}$ results in a vector with the diagonal entries of the argument matrix. On the other hand, $\opn{Dg}$ results in a diagonal matrix with its diagonal entries being the input vector.
\subsection{LB}
\begin{align*}
  \mathcal{L} &= {\mu_{\theta;\xi}^{q}}^{\top}\varphi_{J}^{\top}\left(y -\frac{1}{2}1_{n}\right) + \Tr\left(\varphi_{J}^{\top}\opn{Dg}\left\{\lambda\left(\xi\right) \right\}\varphi_{J}\left(\Sigma_{\theta;\xi}^{q}+\mu_{\theta;\xi}^{q}{\mu_{\theta;\xi}^{q}}^{\top}\right)\right) + 1_{n}^{\top}\Psi\left(\xi\right)\\
  &\quad +\frac{J}{2}\left(\opn{di}\left(\frac{r_{q,\tau}}{2}\right)-\ln\left(\frac{s_{q,\tau}}{2}\right)-\ln\left(2\pi\right)\right) +S_{1}+S_{2}\\
  &\quad +\frac{r_{0,\tau}}{2}\ln\left(\frac{s_{0,\tau}}{2}\right)-\ln\Gamma\left(\frac{r_{0,\tau}}{2}\right) + \frac{r_{0,\tau}}{2}\left(\opn{di}\left(\frac{r_{q,\tau}}{2}\right)-\ln\left(\frac{s_{q,\tau}}{2}\right)\right) - \frac{s_{0,\tau}}{2}\frac{r_{q,\tau}}{s_{q,\tau}} -\ln\frac{\omega_{0}}{2}\\
  &\quad +\frac{r_{q,\tau}}{2} + \ln\Gamma\left(\frac{r_{q,\tau}}{2}\right) -\frac{r_{q,\tau}}{2}\opn{di}\left(\frac{r_{q,\tau}}{2}\right) + \frac{J}{2}\left(1+\ln\left(2\pi\right)\right) + \frac{1}{2}\ln\left|\Sigma_{\theta;\xi}^{q}\right|\\
  &\quad + \frac{1}{2}\left(\ln\left(2\pi{\sigma_{\psi}^{q}}^{2}\right)+1\right)
\end{align*}
\end{document}