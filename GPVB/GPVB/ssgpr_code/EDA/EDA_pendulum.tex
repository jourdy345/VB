\documentclass[11pt]{article}

% This first part of the file is called the PREAMBLE. It includes
% customizations and command definitions. The preamble is everything
% between \documentclass and \begin{document}.

\usepackage[margin=0.6in]{geometry} % set the margins to 1in on all sides
\usepackage{graphicx} % to include figures
\usepackage{amsmath} % great math stuff
\usepackage{amsfonts} % for blackboard bold, etc
\usepackage{amsthm} % better theorem environments
% various theorems, numbered by section
\usepackage{amssymb}
\usepackage[utf8]{inputenc}
\usepackage{booktabs}
\usepackage{array}
\usepackage{courier}
\usepackage[usenames, dvipsnames]{color}
\usepackage{titlesec}
\usepackage{empheq}
\usepackage{tikz}

\newcommand\encircle[1]{%
  \tikz[baseline=(X.base)] 
    \node (X) [draw, shape=circle, inner sep=0] {\strut #1};}
 
% Command "alignedbox{}{}" for a box within an align environment
% Source: http://www.latex-community.org/forum/viewtopic.php?f=46&t=8144
\newlength\dlf  % Define a new measure, dlf
\newcommand\alignedbox[2]{
% Argument #1 = before & if there were no box (lhs)
% Argument #2 = after & if there were no box (rhs)
&  % Alignment sign of the line
{
\settowidth\dlf{$\displaystyle #1$}  
    % The width of \dlf is the width of the lhs, with a displaystyle font
\addtolength\dlf{\fboxsep+\fboxrule}  
    % Add to it the distance to the box, and the width of the line of the box
\hspace{-\dlf}  
    % Move everything dlf units to the left, so that & #1 #2 is aligned under #1 & #2
\boxed{#1 #2}
    % Put a box around lhs and rhs
}
}


\newtheorem{thm}{Theorem}[section]
\newtheorem{lem}[thm]{Lemma}
\newtheorem{prop}[thm]{Proposition}
\newtheorem{cor}[thm]{Corollary}
\newtheorem{conj}[thm]{Conjecture}

\setcounter{secnumdepth}{4}

\titleformat{\paragraph}
{\normalfont\normalsize\bfseries}{\theparagraph}{1em}{}
\titlespacing*{\paragraph}
{0pt}{3.25ex plus 1ex minus .2ex}{1.5ex plus .2ex}

\definecolor{myblue}{RGB}{72, 165, 226}
\definecolor{myorange}{RGB}{222, 141, 8}

\setlength{\heavyrulewidth}{1.5pt}
\setlength{\abovetopsep}{4pt}


\DeclareMathOperator{\id}{id}
\DeclareMathOperator{\argmin}{\arg\!\min}
\DeclareMathOperator{\Tr}{Tr}

\newcommand{\bd}[1]{\mathbf{#1}} % for bolding symbols
\newcommand{\RR}{\mathbb{R}} % for Real numbers
\newcommand{\ZZ}{\mathbb{Z}} % for Integers
\newcommand{\col}[1]{\left[\begin{matrix} #1 \end{matrix} \right]}
\newcommand{\comb}[2]{\binom{#1^2 + #2^2}{#1+#2}}
\newcommand{\bs}{\boldsymbol}
\newcommand{\opn}{\operatorname}
\begin{document}
\nocite{*}

\title{Exploratory data analysis}

\author{Daeyoung Lim\thanks{Prof. Taeryon Choi} \\
Department of Statistics \\
Korea University}

\maketitle
\section{Pendulum Data}
\subsection{Data Description}
The pendulum data obtained from the paper of David Nott consists of 9 covariates and one target variable. As described in the original paper, it is a simulated data of mechanical pendulum, covariates of which are different parameters of the system and the target variable is the angular velocity.
\subsection{Angular Velocity}
According to Wikipedia, an angular velocity is \textit{``defined as the rate of change of angular displacement and is a vector quantity (more precisely, a pseudovector) which specifies the angular speed of an object and the axis about which the object is rotating''}. The unit is radians per second.
\section{SSGP}
  For a short summary of Lazaro Gredilla's SSGP, the paper comes up with a decomposition of the function into
  \begin{equation}
    f\left(\mathbf{x}\right) = \sum_{r=1}^{m}a_{r}\cos\left(2\pi \mathbf{s}_{r}^{\top}\mathbf{x}\right) + b_{r}\sin\left(2\pi\mathbf{s}_{r}^{\top}\mathbf{x}\right)
  \end{equation}
  where $a_{r}\sim \mathcal{N}\left(0, m^{-1}\sigma_{0}^{2}\right)$, $b_{r} \sim \mathcal{N}\left(0,m^{-1}\sigma_{0}^{2}\right)$. Then, by transforming $\mathbf{x}$ into
  \begin{equation}
    \phi\left(\mathbf{x}\right) = \begin{bmatrix} \cos\left(2\pi\mathbf{s}_{1}^{\top}\mathbf{x}\right) & \sin\left(2\pi\mathbf{s}_{1}^{\top}\mathbf{x}\right) & \cdots & \cos\left(2\pi\mathbf{s}_{m}^{\top}\mathbf{x}\right) & \sin\left(2\pi\mathbf{s}_{m}^{\top}\mathbf{x}\right) \end{bmatrix},
  \end{equation}
  the end result becomes similar to linear Gaussian process regression model:
  \begin{align}
    \opn{E}\left(\mathbf{y}_{*}\right) &= \phi\left(\mathbf{x}_{*}\right)^{\top}A^{-1}\mathbf{\Phi}\mathbf{y}\\
    \opn{Var}\left(\mathbf{y}_{*}\right) &= \sigma_{n}^{2}+\sigma_{n}^{2}\phi\left(\mathbf{x}_{*}\right)^{\top}A^{-1}\phi\left(\mathbf{x}_{*}\right).
  \end{align}
  The author suggests learning the parameters via optimizing the log-marginal likelihood:
  \begin{equation}
    \log p\left(\mathbf{y}|\theta\right) = -\frac{1}{2\sigma_{n}^{2}}\left(\mathbf{y}^{\top}\mathbf{y}-\mathbf{y}^{\top}\mathbf{\Phi}^{\top}A^{-1}\mathbf{\Phi y}\right)-\frac{1}{2}\log \left|A\right| +m\log\frac{m\sigma_{m}^{2}}{\sigma_{0}^{2}}-\frac{n}{2}\log\left(2\pi\sigma_{n}^{2}\right)
  \end{equation}
  by means of the conjugate gradient method.
\subsection{Sidenote}
  The predictive mean of Gaussian process regression model is
  \begin{align}
    p\left(f_{*}|\mathbf{x}_{*}, X, \mathbf{y}\right) &= \int p\left(f_{*}|\mathbf{x}_{*}, \mathbf{w}\right)p\left(\mathbf{w}|X,\mathbf{y}\right)\,d\mathbf{w} = \int \mathbf{x}_{*}^{\top}\mathbf{w}p\left(\mathbf{w}|X,\mathbf{y}\right)\,d\mathbf{w}\\
    &=\mathcal{N}\left(\frac{1}{\sigma_{n}^{2}}\mathbf{x}_{*}^{\top}A^{-1}X\mathbf{y}, \mathbf{x}_{*}^{\top}A^{-1}\mathbf{x}_{*}\right).
  \end{align}
  The posterior distribution of the weights $\mathbf{w}$ is
  \begin{equation}
    p\left(\mathbf{w}|X,\mathbf{y}\right) \sim \mathcal{N}\left(\overline{\mathbf{w}}=\frac{1}{\sigma_{n}^{2}}A^{-1}X\mathbf{y}, A^{-1}\right)
  \end{equation}
  where $A = \sigma_{n}^{-2}XX^{\top}+\Sigma_{p}^{-1}$. Therefore, this is the reason why we can plug in the test data $\overset{\sim}{X}$ into the predictive distribution's $\mathbf{x}_{*}$ to get the fitted values of the unknown function, $\hat{f}$. (In the case of sparse spectrum decomposition, the design matrix is no longer $X$ but rather $\Phi$.)
\section{VA for partially linear additive models}
The paper suggests a model of the form
\begin{equation}
  y_{i} = \mu + \sum_{j=1}^{p}f_{j}\left(x_{ij}\right) + \epsilon_{i}, \quad i = 1, \ldots , n.
\end{equation}
\begin{itemize}
  \item In the authors' exact words, \textit{for simplicity of exposition}, they assumed all covariates to be continuous with support $\left[0, 1\right]$.
  \item Discrete variables are put into the linear part.
  \item Impose $\opn{E}\left(f_{j}\left(x_{j}\right)\right)=0$ constraint to achieve identiability.
  \item Transform the basis functions in such a way that they are orthogonal to the linear basis functions.
  \item Essentially VB for parameter estimation but offered some degree of freedom between Monte Carlo estimation and Laplace approximation for the intractable integration terms.
\end{itemize}
The real data analysis section of the original paper mentions that the crime data in \textsf{R} package \{\textsf{Ecdat}\} was used, which has 630 observations and 22 variables. 
\subsection{BPLAM code review}
  \begin{itemize}
  \item Since one of the purposes of this paper was to propose a novel way of selecting variables using variational approximation, the code assumes there would be more than one covariate. The code crashes with only \texttt{x} of one column.
  \item The code still worked even though one of the covariates was outside of the range $\left[0,1\right]$. 
  \end{itemize}
\section{Kneib VA}
Model:
\begin{equation}
  y_{i} = \mathbf{x}^{\top}\bs{\beta}+f_{1}\left(\text{herdsize}_{i}\right)+f_{2}\left(\text{capital}_{i}\right)+f_{\text{spat}}\left(\text{county}_{i}\right) + \epsilon_{i}.
\end{equation}
Code doesn't work...
\section{To be done}
\begin{itemize}
  \item Fix Kneib's code
  \item EDA Miguel's datasets:
    \begin{itemize}
      \item \texttt{elevators} data
      \item \texttt{pumadyn32nm} data
      \item \texttt{pendulum} data
      \item \texttt{kin40k} data
      \item \texttt{pol} data
    \end{itemize}
    and BPLAM real data set: \textsf{Ecdat} crime data.
  \item Run BSAR VB/MCMC (no restriction/shape restriction), SSGP, spline VB, Kneib VB
    \begin{itemize}
      \item real data sets (listed above)
      \item simulated data: 3 models in Kneib's paper, 1 model in BPLAM etc
    \end{itemize}
    and compare RMSE, code execution time etc.
\end{itemize}
\section{Datasets}
\subsection{Pole Telecomm and Elevators}
  According to Lazaro Gredilla, the data sets are taken from \texttt{http://www.liaad.up.pt/~ltorgo/Regression/DataSets.html}. 
\end{document}